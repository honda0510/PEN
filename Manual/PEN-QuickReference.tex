¥documentclass[11pt,a4j]{jarticle}
¥usepackage{ascmac}
¥usepackage{times}
¥usepackage{graphicx}

¥setlength¥textwidth{17.0cm}
¥setlength¥textheight{26.0cm}

%ページ上の余白
¥setlength¥topmargin{-2.54cm}

%奇数ページ左の余白
¥setlength¥oddsidemargin{-0.54cm}

%偶数ページ左の余白
¥setlength¥evensidemargin{-0.54cm}

¥begin{document}
¥noindent
¥begin{flushright}
{¥small	2007.01.08}
¥end{flushright}
¥begin{center}
{¥large¥bf PENクイックリファレンス}
¥end{center}

¥begin{enumerate} ¥itemsep 0pt  ¥parskip 0pt
¥item 基本画面 ¥¥
¥begin{figure}[htbp]
  ¥begin{center}
    ¥vspace{-0.5cm}
    ¥includegraphics[width=12.0cm]{./eps/pen001.eps}
    ¥vspace{-1.0cm}
  ¥end{center}
¥end{figure}
  
¥begin{enumerate} ¥itemsep 0pt  ¥parskip 0pt
¥item 編集画面 ¥¥
プログラムのソースコードを入力するエリアです。 

¥vspace{0.2cm}

¥item コンソール画面 ¥¥
プログラム中の出力はこの画面に表示されます。
また、入力もこの画面から行います。 ¥¥
タブの操作により表示方法の異なる「実行画面」と「履歴」に切り替えられます。 
¥begin{itemize} ¥itemsep 0pt  ¥parskip 0pt
¥item 実行画面 : 実行中もしくは実行直後のコンソールが表示されます。
¥item 履歴     : 今までの実行した結果の全てが表示されます。
¥end{itemize}

¥vspace{0.2cm}

¥item 変数表示画面 ¥¥
データ型、変数名と、変数に代入されている値が表示されます。 ¥¥
プログラム実行時に変数の値の変化を観察することができます。 

¥vspace{0.2cm}

¥item プログラム入力支援ボタン ¥¥
プログラムの入力を補助するためのボタンです。 ¥¥
¥begin{minipage}{12.0cm} ¥begin{itembox}[l]{「もし〜そうでなければ」のボタンで入力されるコード} ¥begin{verbatim}
もし ≪条件式≫ ならば
  |
を実行し,そうでなければ
  |
を実行する
¥end{verbatim} ¥end{itembox} ¥end{minipage} ¥¥
≪条件式≫など≪≫に囲まれた部分にカーソルを移動すると≪≫の部分全体が選択され、 ¥¥
そこに書くべき式などに簡単に書き換えることができます。

¥vspace{0.2cm}

¥item GraphicWindow ¥¥
図形描画の出力はこのウィンドウに表示されます。 ¥¥
描画関連の命令は `xDNCL-Draw.pdf' をご覧ください。
¥end{enumerate}

¥newpage

¥item 実行時の画面 
¥begin{figure}[htbp]
  ¥vspace{-0.3cm}
  ¥begin{center}
    ¥includegraphics[width=12.0cm]{./eps/pen002.eps}
    ¥vspace{-1.0cm}
  ¥end{center}
¥end{figure}

¥begin{enumerate} ¥itemsep 0pt  ¥parskip 0pt
¥item 実行制御ボタン
¥begin{table}[htbp]
  ¥vspace{-0.3cm}
  ¥begin{center}
    ¥begin{tabular}{l|p{11.0cm}} ¥hline
	状態表		& 意味 ¥¥ ¥hline
	実行		&
                  プログラムを実行する場合はこのボタンを押します。
                  プログラム実行中は「一時停止」ボタンに変化します。
                  実行する速度は「実行速度調整スライダー」で変化させることができます。 ¥¥ ¥hline
    始めから実行&
                  実行ボタンと同じ。 ¥¥ ¥hline
    一時停止	&
                  プログラム実行中に押すと実行を一時停止することができます。
                  一時停止時は、「再開」のボタンに変化します。 ¥¥ ¥hline
    再開		&
                  一時停止状態から実行を再開したい場合に使用します。
                  再開後は、「一時停止」のボタンに変化します。 ¥¥ ¥hline
	一行実行	&
	              「実行箇所マーカー」のある行を実行し、その後、一時停止状態になります。 ¥¥ ¥hline
	始めに戻る	&
	              プログラムの実行を取り止め、プログラムの最初に戻ります。 ¥¥ ¥hline
    ¥end{tabular}
  ¥end{center}
  ¥vspace{-0.7cm}
¥end{table}

¥vspace{0.2cm}

¥item 実行速度調整スライダー ¥¥
プログラムの実行速度を変更するためのスライダーです。 ¥¥
バーをゲージの左側に移動すると実行速度が速くなり、右側に移動すると遅くなります。

¥vspace{0.2cm}

¥item 状態表示
¥begin{table}[htbp]
  ¥vspace{-0.3cm}
  ¥begin{center}
    ¥begin{tabular}{l|p{11.0cm}} ¥hline
	状態表		& 意味 ¥¥ ¥hline
	実行待ち	& プログラムを実行していない状態。
				プログラムの編集が可能。 ¥¥ ¥hline
	実行中		& 通常にプログラムを実行している状態。 ¥¥ ¥hline
	一時停止中	& 「一時停止」や「一行実行」によって実行
			  	が停止されている状態。 ¥¥ ¥hline
	入力待ち	& プログラム内の入力文による入力待ち状態。
				入力はコンソール画面に行う。 ¥¥ ¥hline
	実行終了	& プログラムの実行が終了した状態。
			 	再度、プログラムを実行する場合は
				「始めから実行」で行う。
				また状態を「実行待ち」に初期化する
				には「始めに戻る」で行う。 ¥¥ ¥hline
    ¥end{tabular}
  ¥end{center}
  ¥vspace{-0.7cm}
¥end{table}

¥vspace{0.2cm}

¥item 実行箇所マーカー ¥¥
これから実行する行を指し示しています。 ¥¥
この場面では16行目を実行する直前の状態になります
¥end{enumerate}
¥end{enumerate}

¥end{document} 
