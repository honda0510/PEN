%
%PEN INSTRUCTION
%
¥documentclass[10pt,a4j]{jarticle}
%¥pagestyle{empty}
¥usepackage{amsmath}
¥usepackage{amssymb}
¥usepackage{epsfig}
¥usepackage{color}
%¥usepackage{txfonts}

%¥setlength{¥topmargin}{-0mm}
%¥setlength{¥oddsidemargin}{-0mm}
%¥setlength{¥textheight}{252mm}
%¥setlength{¥textwidth}{180mm}

%---------- 箇条書き -----------

¥newenvironment{itemize2}%  
{%
   ¥begin{list}{$¥bullet$¥ ¥ }% 見出し記号/直後の空白を調節
   {%
      ¥setlength{¥itemindent}{0pt}
      ¥setlength{¥leftmargin}{3zw}%  左のインデント
      ¥setlength{¥rightmargin}{0zw}% 右のインデント
      ¥setlength{¥labelsep}{0zw}%    黒丸と説明文の間
      ¥setlength{¥labelwidth}{3zw}%  ラベルの幅
      ¥setlength{¥itemsep}{0em}%     項目ごとの改行幅
      ¥setlength{¥parsep}{0em}%      段落での改行幅
      ¥setlength{¥listparindent}{0zw}% 段落での一字下り
   }
}{%
   ¥end{list}%
}
%---------- 番号つき箇条書き -----------

¥newcounter{enum2}
¥newenvironment{enumerate2}{%
   ¥begin{list}%
   {%
      ¥arabic{enum2}.¥ ¥,%  見出し記号/直後の空白を調節
   }%
   {%
      ¥usecounter{enum2}
      ¥setlength{¥itemindent}{0zw}%  ここは 0 に固定
      ¥setlength{¥leftmargin}{3zw}%  左のインデント
      ¥setlength{¥rightmargin}{0zw}% 右のインデント
      ¥setlength{¥labelsep}{0zw}%    黒丸と説明文の間
      ¥setlength{¥labelwidth}{3zw}%  ラベルの幅
      ¥setlength{¥itemsep}{0em}%     項目ごとの改行幅
      ¥setlength{¥parsep}{0em}%      段落での改行幅
      ¥setlength{¥listparindent}{0zw}% 段落での一字下り
   }
}{%
   ¥end{list}%
}
¥setlength{¥topsep}{0pt}
¥setlength{¥itemsep}{10pt}
¥setlength{¥parsep}{0mm}
¥setlength{¥itemindent}{50mm}

¥labelsep 20mm
¥labelwidth 50mm
¥leftmargin 100mm
¥newcommand{¥ExerciseTitle}[1]{%
  ¥begin{center}¥Large
    ¥textbf{PEN マニュアル Q ¥& A} ¥qquad ¥texttt{#1}
  ¥end{center}%
}


¥renewcommand{¥baselinestretch}{1.0}

¥begin{document}

¥noindent
¥begin{flushright}
{¥small	2009/02/16}
¥end{flushright}
¥begin{center}
¥begin{LARGE}
{¥bf{PENの動作を制御する}}
¥end{LARGE}
¥end{center}

Property.ini 内で指定できる設定項目は以下の通り。
パラメータは起動時に読み込まれる。

¥section{変数宣言の要/不要の設定}

変数宣言の要/不要を設定する。

¥begin{quotation}

¥noindent [使用例]¥¥
~~~~{¥bf{executer.var.declaration=0}}¥¥
¥ ¥¥ 
設定値は以下の通り(デフォルト値は0)。¥¥

¥begin{tabular}{c|l||cl}
¥hline
¥multicolumn{2}{c||}{設定値} & ¥multicolumn{2}{c}{変数宣言がない場合の出力}¥¥
¥hline
0 & 変数宣言必須 &  & エラー ¥¥
¥hline
1 & 変数宣言不要 &  & 警告表示 ¥¥
¥hline
2 & 変数宣言不要 &  & エラーと警告表示なし ¥¥
¥hline
¥end{tabular}

¥end{quotation}

¥section{配列の添字の範囲を設定}
配列を宣言して変数を確保するとき、添字の範囲を決めることができる。

¥begin{quotation}
¥noindent [使用例]¥¥
~~~~{¥bf{executer.array.origin=0}}¥¥
¥ ¥¥ 
設定値は以下の通り。¥¥

¥begin{tabular}{c|l}
¥hline
設定値  & 配列 X[n] を宣言した場合 ¥¥
¥hline
0       & X[0],...,X[n]  の要素を確保する (デフォルト値) ¥¥
¥hline
1       & X[0],...,X[n-1]の要素を確保する ¥¥
¥hline
2       & X[1],...,X[n]  の要素を確保する ¥¥
¥hline
¥end{tabular}

%¥begin{tabular}{ccl}
%設定値 & : & 配列 X[n] を宣言した場合 ¥¥
%0      & : & X[0],...,X[n]  の要素を確保する (デフォルト値) ¥¥
%1      & : & X[0],...,X[n-1]の要素を確保する ¥¥
%2      & : & X[1],...,X[n]  の要素を確保する ¥¥
%¥end{tabular}

¥end{quotation}

¥section{描画ウィンドウの原点の設定}

描画ウィンドウで座標を指定するときの原点を設定する。

¥begin{quotation}
¥noindent [使用例]¥¥
~~~~{¥bf{executer.graphic.origin=0}}¥¥
¥ ¥¥
設定値は以下の通り。¥¥

¥begin{tabular}{c|l}
¥hline
0 & 左上を原点にする (デフォルト値) ¥¥
¥hline
1 & 左下を原点にする ¥¥
¥hline
¥end{tabular}
¥end{quotation}

%¥begin{tabular}{ccl}
%0 & : & 左上を原点にする (デフォルト値) ¥¥
%1 & : & 左下を原点にする ¥¥
%¥end{tabular}
%¥end{quotation}

¥section{入力支援ボタンの定義を別ファイルで行う}

入力支援ボタンの設定を別ファイルに記述しておくことができる。
定義ファイルまでのパスを設定値に記述しておくと、
PENはその定義ファイルを用いて起動する。

通常は、このファイルを指定せずにPENを起動することができる。

¥begin{quotation}
¥noindent [使用例]¥¥
~~~~{¥bf{ pen.button.path=./ButtonList.ini}}

¥end{quotation}

¥section{フォントサイズを大きくして起動する設定}

起動時のフォントサイズを大きくすることができる。

¥begin{quotation}
¥noindent [使用例]¥¥
~~~~{¥bf{pen.teacher.flag=0}}¥¥
¥ ¥¥
設定値は以下の通り。¥¥

¥begin{tabular}{c|l}
¥hline
0 & 通常の大きさ (デフォルト値)¥¥
¥hline
1 & フォントサイズを拡大して起動 ¥¥
¥hline
¥end{tabular}¥¥

%¥begin{tabular}{ccl}
%0 & : & 通常起動 ¥¥
%1 & : & フォントサイズを大きくして起動 ¥¥
%¥end{tabular}

図~¥ref{fig:font01}は拡大したサイズ、図~¥ref{fig:font02}は標準サイズでの表示。

¥end{quotation}

¥begin{figure}[!h]
¥begin{minipage}{50zw}
¥begin{center}
¥epsfig{file=./eps/pen_font_large.ps, width=5.5in}
¥caption{$¥!¥!¥!¥!$¥colorbox{white}{{¥textcolor{white}{:}}}拡大サイズ }
¥label{fig:font01}
¥end{center}
¥end{minipage}
¥ ¥¥
¥ ¥¥
¥begin{minipage}{50zw}
¥begin{center}
¥epsfig{file=./eps/pen_font_small.ps, width=5.5in}
¥caption{$¥!¥!¥!¥!$¥colorbox{white}{{¥textcolor{white}{:}}}標準サイズ}
¥label{fig:font02}
¥end{center}
¥end{minipage}

¥end{figure}


¥section{デバックモードでの起動}
デバッグモードを利用することができる。
デバッグ結果(構文解析されたツリー)はコンソールに出力される。

¥begin{quotation}
¥noindent [使用例] デバッグモードでの起動方法¥¥
~~~~{¥bf{pen.debug.flag=1}}¥¥
~~~~と設定し、コマンドプロンプト上で、
~~~~> java -jar PEN.jar
~~~~と入力し起動する。

¥ ¥¥
設定値は以下の通り。¥¥

¥begin{tabular}{c|l}
¥hline
0 & 通常起動 (デフォルト値) ¥¥
¥hline
1 & debugモードで起動 ¥¥
¥hline
¥end{tabular}

%¥begin{tabular}{ccl}
%0 & : & 通常起動 (デフォルト値) ¥¥
%1 & : & debugモードで起動 ¥¥
%¥end{tabular}

¥end{quotation}

¥section{エラー時のダンプ動作設定}
xDNCLの実行時にエラーが発生した場合、
dumpファイルを出力させるように設定することができる。
また、出力先の指定もできる。

¥begin{quotation}
¥noindent [使用例]¥¥
~~~~{¥bf{pen.dump.flag=0}}¥¥
¥ ¥¥
設定値は以下の通り。¥¥

¥begin{tabular}{c|l}
¥hline
0 & dumpファイルを生成しない (デフォルト値) ¥¥
¥hline
1 & dumpファイルを生成する ¥¥
¥hline
¥end{tabular}

%¥begin{tabular}{ccl}
%0 & : & dumpファイルを生成しない (デフォルト値) ¥¥
%1 & : & dumpファイルを生成する ¥¥
%¥end{tabular}

¥end{quotation}

¥subsection{dumpファイルの暫定保存先を指定}
dumpファイルの作成を指定した場合、ファイルの一時保存先を指定することができる。
¥begin{quotation}
¥begin{tabular}{l}
出力先を設定しなければ HOMEディレクトリへ出力される。¥¥
(Windowsの場合は、C:¥Document~and~Settings¥ ユーザー名) ¥¥
¥ ¥¥
出力先の指定方法は、¥¥
{¥bf{pen.dump.tempdir=一時保存先}} ¥¥
¥end{tabular}
¥end{quotation}

¥subsection{一時保存したdumpファイルの最終保存先を指定}
%# (ログファイル保存先にdumpファイルを移動)
dumpファイルの作成を指定した場合、ファイルの最終保存先を指定することができる。
¥begin{quotation}

¥begin{tabular}{l}
出力先を設定しなければ HOMEディレクトリへ出力される。¥¥
(Windowsの場合は、C:¥Document¥ and¥ Settings¥ユーザー名) ¥¥
出力されるファイル名は、「ユーザー名-コンピュータ名-数字.log」となる。¥¥
¥ ¥¥
出力先の指定方法は、¥¥
{¥bf{pen.dump.destdir=ログファイル保存先}} ¥¥
¥end{tabular}

¥end{quotation}

¥end{document}


