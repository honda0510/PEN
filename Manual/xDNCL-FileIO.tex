%
%PEN INSTRUCTION
%
¥documentclass[10pt,a4j]{jarticle}
%¥pagestyle{empty}
¥usepackage{amsmath}
¥usepackage{amssymb}
%¥usepackage{txfonts}

%¥setlength{¥topmargin}{-0mm}
%¥setlength{¥oddsidemargin}{-0mm}
%¥setlength{¥textheight}{252mm}
%¥setlength{¥textwidth}{180mm}

%---------- 箇条書き -----------

¥newenvironment{itemize2}%  
{%
   ¥begin{list}{$¥bullet$¥ ¥ }% 見出し記号/直後の空白を調節
   {%
      ¥setlength{¥itemindent}{0pt}
      ¥setlength{¥leftmargin}{3zw}%  左のインデント
      ¥setlength{¥rightmargin}{0zw}% 右のインデント
      ¥setlength{¥labelsep}{0zw}%    黒丸と説明文の間
      ¥setlength{¥labelwidth}{3zw}%  ラベルの幅
      ¥setlength{¥itemsep}{0em}%     項目ごとの改行幅
      ¥setlength{¥parsep}{0em}%      段落での改行幅
      ¥setlength{¥listparindent}{0zw}% 段落での一字下り
   }
}{%
   ¥end{list}%
}
%---------- 番号つき箇条書き -----------

¥newcounter{enum2}
¥newenvironment{enumerate2}{%
   ¥begin{list}%
   {%
      ¥arabic{enum2}.¥ ¥,%  見出し記号/直後の空白を調節
   }%
   {%
      ¥usecounter{enum2}
      ¥setlength{¥itemindent}{0zw}%  ここは 0 に固定
      ¥setlength{¥leftmargin}{3zw}%  左のインデント
      ¥setlength{¥rightmargin}{0zw}% 右のインデント
      ¥setlength{¥labelsep}{0zw}%    黒丸と説明文の間
      ¥setlength{¥labelwidth}{3zw}%  ラベルの幅
      ¥setlength{¥itemsep}{0em}%     項目ごとの改行幅
      ¥setlength{¥parsep}{0em}%      段落での改行幅
      ¥setlength{¥listparindent}{0zw}% 段落での一字下り
   }
}{%
   ¥end{list}%
}
¥setlength{¥topsep}{0pt}
¥setlength{¥itemsep}{10pt}
¥setlength{¥parsep}{0mm}
¥setlength{¥itemindent}{50mm}

%¥labelsep 20mm
%¥labelwidth 50mm
%¥leftmargin 100mm
¥newcommand{¥ExerciseTitle}[1]{%
  ¥begin{center}¥Large
    ¥textbf{PEN マニュアル Q ¥& A} ¥qquad ¥texttt{#1}
  ¥end{center}%
}

¥renewcommand{¥baselinestretch}{1.0}

¥begin{document}

¥noindent
¥begin{flushright}
{¥small	2008/10/14}
¥end{flushright}
¥begin{center}
¥begin{LARGE}
{¥bf{ファイル入出力のための関数群¥¥
¥ ¥¥}}
¥end{LARGE}
¥end{center}

  プログラムの中から、ファイル内のデータを読み書きするための関数群を紹介する。¥¥
%¥ ¥¥
%  ファイルを読み書きするためには、まず、当該ファイルを開く(オープン
%する)必要がある。また、ファイルの読み書きが終了したとき、ファイルを
%閉じなければならない(クローズ)。¥¥
%¥ ¥¥
%  以下、各関数について説明する。¥¥
¥ ¥¥
(1) ファイルのオープン/クローズ関数 
¥begin{enumerate2}
¥item {¥bf{整数 fd ← openr(文字列 filepath)}} ¥¥
%   整数  fd,	   文字列  filepath ¥¥
	filepathで指定したファイルを、データの読み込みのために開く。
	fdはファイル識別子であり、以後、fdを指定して、ファイルからの読み込み等を行う。

¥item {¥bf{整数 fd ← openw(文字列 filepath)}} ¥¥
%   整数  fd,	   文字列  filepath ¥¥
	filepathで指定したファイルを、データの書き込みのために開く。
	指定したファイルが既に存在しているならば、
        そのファイルはこれから書き出すデータで置き換えられる。

¥item {¥bf{整数 fd ← opena(文字列 filepath)}} ¥¥
%   整数  fd,	   文字列  filepath ¥¥
	filepathで指定したファイルを、データの追記書込みのために開く。
	指定したファイルが既に存在しているならば、
        そのファイルの続きとして、これから書き出すデータを追記する。

¥item {¥bf{close(整数 fd)}} ¥¥
%   整数  fd ¥¥
	fdで指定したファイルを閉じる。
¥end{enumerate2}

(2) ファイルの読み書きのための関数 
¥begin{enumerate2}
¥item {¥bf{文字列 str ← getstr(整数 fd, 整数 n)}} ¥¥
%   文字列  str,   整数  fd,  n ¥¥
	fdで指定したファイルから、n文字読み込んで その文字列を返す。
        ただし、改行コード(Return, Linefeed)もそれぞれ1文字と数える。

¥item {¥bf{文字列 str ← getline(整数 fd)}} ¥¥
%   文字列  str,   整数  fd ¥¥
	fdで指定したファイルから、1行読み込んで その文字列を返す。

¥item {¥bf{putstr(整数 fd, 文字列 str)}} ¥¥
%   整数  fd,     文字列   str ¥¥
	fdで指定したファイルに、strの文字列を書き出す。

¥item {¥bf{putline(整数 fd, 文字列 str)}} ¥¥
%   整数  fd,     文字列   str ¥¥
	fdで指定したファイルに、strの文字列を書き出す。
	putstr()は、文字列だけを書き出すのに比べて、
	putline()は、文字列の後に改行コードも書き出す。
¥end{enumerate2}

(3) ファイル処理のための補助関数 
¥begin{enumerate2}
¥item {¥bf{文字列 str ← isfile(文字列 filepath)}} ¥¥
%   文字列  str,  文字列  filepath ¥¥
	filepathで指定したファイルが存在するかどうかを調べる。
	ファイルが存在する場合 "true"、 
        存在しない場合"false"の文字列が返される。

¥item {¥bf{rename(文字列 filepath1, 文字列 filepath2)}} ¥¥
%   文字列  filepath1,  filepath2 ¥¥
	filepath1で指定したファイルの名前を、filepath2に置き換える。

¥item {¥bf{flush(整数 fd)}} ¥¥
%   整数  fd ¥¥
	fdで指定されるファイルに書き込み途中のデータを強制的に書込む。
        (ファイルは通常バッファリングされるので、
	この関数を呼び出すことで、バッファ内容を吐き出す) 

¥item {¥bf{remove(文字列 filepath)}} ¥¥
%   文字列  filepath ¥¥
	filepathで指定したファイルを削除する。
¥end{enumerate2}

¥end{document}

